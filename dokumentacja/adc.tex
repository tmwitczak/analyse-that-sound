\subsection{ADC}

\paragraph{Opis funkcjonalności}

    Konwerter wykonujący 12-bitową aproksymację sygnału analogowego do
    postaci cyfrowej.

\paragraph{Konfiguracja}

    Po resecie domyślnie ADC jest wyłączony.

    W celu dostarczenia zasilania do urządzenia, najpierw w rejestrze
    PCONP ustawiamy bit 12 na 1. Konwerter nie jest jeszcze w stanie
    operacyjnym.

    Aby ustawić dzielnik zegara na 4, wpisujemy do bitów 24 i 25
    rejestru PCLKSEL0 kolejno wartości 0 i 0.

    Następnie ustawiamy funkcję pinu P0.23 na AD0.0 poprzez wpisanie
    do bitów 14 i 15 rejestru PINSEL1 kolejno wartości 0 i 0, a potem
    do bitu 14 wartości 1.

    Kolejnym krokiem jest reset konwertera poprzez wyzerowanie rejestru
    AD0CR, a potem faktyczne włączenie urządzenia poprzez ustawienie
    bitu 21 na wartość 1.

    Wybieramy, z których pinów (AD0.7:0) będą próbkowane i konwertowane
    wartości sygnału. W celu wykonywania konwersji na ustawionym
    wcześniej pinie P0.23, w rejestrze AD0CR ustawiamy bit 0 na wartość
    1.

    Zegar dla konwertera nie powinien przekraczać wartości
    częstotliwości 13 MHz, jednakże częstotliwość dostarczana
    od procesora, po uwzględnieniu dzielenia przez 4, to 25 MHz.
    W celu ustawienia dodatkowego dzielnika zegara dla ADC na wartość 2
    i osiągnięcia w ten sposób taktowania zgodnego z wymaganym
    przedziałem, do bitu 8 rejestru AD0CR wpisujemy 1. 

    Uruchomianie przerwań po wykonaniu konwersji uzyskujemy poprzez
    ustawienie na wartość 1 bitu 8 rejestru AD0INTEN.

    Częstotliwość próbkowania sygnału określamy przy użyciu TIMER0.
    Powiązanie tych dwóch urządzeń następuje przy wykorzystaniu pinu
    MAT0.1. W tym celu najpierw jawnie wyłączamy wykorzystanie trybu
    BURST konwertera poprzez ustawienie bitu 16 rejestru AD0CR na
    wartość 0. Aby konwersja została uruchomiona w momencie wystąpienia
    narastającego zbocza, ustawiamy w rejestrze AD0CR bity 26, 25
    i 24 kolejno na wartości 1, 0 i 0.

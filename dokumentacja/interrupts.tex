\subsection{Obsługa przerwań} \label{interrupts}
Przerwania są podstawowym mechanizmem występującym w każdym procesorze. Ogólna zasada działania jest taka, że w pamięci procesora przechowywany jest wektor przerwań, czyli tablica odwzorowująca numery przerwań na adresy funkcji, zwanych procedurami obsługi przerwań. Kiedy dojdzie do pewnych zdarzeń w urządzeniu, może zostać wygenerowane przerwanie. Główny wątek programu zostaje przerwany i następuje skok do procedury obsługi danego przerwania (każde ma swój numer), której to adres jest uprzednio odczytywany ze wspomnianego wcześniej wektora przerwań. Programista może napisać swoje własne procedury obsługi przerwań i umieścić ich adresy w tym wektorze. W przypadku wykorzystywanego układu istnieje także możliwość nadawać priorytety poszczególnym przerwaniom. Przerwania mogą być także generowane programowo.

W stworzonym programie wykorzystywane są tylko 2 z 35 dostępnych w mikrokontrolerze przerwań. Pierwsze z nich jest przerwaniem ADC, do którego dochodzi po zakończeniu pomiaru (patrz \ref{adc}). Drugie przerwanie jest generowane kiedy dojdzie do zdarzenia Compare Match 1 w Timer 1 (patrz \ref{timer}). Aby przerwanie było generowane musi być ono odblokowane w dwóch miejscach:
\begin{itemize}
    \item rejestry kontrolne danego urządzenia peryferyjnego odpowiedzialnego za generowanie przerwania
    \item rejestry systemowego kontrolera przerwań (NVIC)
\end{itemize}
Konfiguracja rejestrów ADC i Timer 1 zostały opisane w \ref{timer} i \ref{adc}. Aby odblokować wykorzystywane przerwania w NVIC'u wykonane zostały następujące kroki:
\begin{enumerate}
    \item ustawienie bitu 2 rejestru ISER0 aby włączyć przerwania Timera 1
    \item ustawienie bitu 22 rejestru ISER0 aby włączyć przerwania ADC
\end{enumerate}

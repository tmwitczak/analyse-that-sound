\subsection{SSP}

\subsubsection{Opis funkcjonalności}

    Kontroler synchronicznego portu szeregowego zdolny do operacji
    na SPI.

\subsubsection{Konfiguracja}

    Konfigurację SPI rozpoczynamy od ustawiania pinów.
    Piny P0.7, P0.8, P0.9 ustawiamy kolejno na funkcje SCK1,
    MISO1 i MOSI1 poprzez pisanie wartości 2 do rejestru
    PINSEL0 dla bitów (kolejno dla pinów) 15:14, 17:16
    i 19:18. Analogicznie ustawiamy wartości 0 do rejestru
    PINMODE0 i wartości 0 do PINMODE0_OD. Następnie dla P2.2
    ustawiamy funkcję pinu GPIO i analogicznie MODE i 'open drain'
    jak wyżej.

    Ustawiamy zasilanie urządzenia poprzez wpisanie wartości 1
    do bitu 10 rejestru PCONP.

    Potem ustawiamy częstotliwość zegara dla SSP1 równą 1 MHz
    poprzez.... 

    Następnie piszemy do rejestru CR0 od SSP1

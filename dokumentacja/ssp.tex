\subsection{SSP} \label{ssp}
SSP jest kontrolerem synchronicznego portu szeregowego, który między innymi może operować na SPI. Wykorzystywane w projekcie urządzenie posiada dwa takie interfejsy (SSP0 i SSP1). Każdy z nich posiada swoje własne rejestry (CR0, CR1, ...) Interfejs SSP1 jest wykorzystywany do komunikacji z wyświetlaczem OLED (patrz \ref{oled}). Konfiguracja przebiega w następujący sposób:
\begin{enumerate}
    \item Podłączenie pinów P0.7, P0.8 i P0.9 do wyjść SCK1, MISO1 i MOSI1 interfejsu SSP1 poprzez wpisanie wartości 2 do rejestru PINSEL0, do par bitów 14:15, 16:17 i 18:19
    \item Włączenie zasilania dla SSP1 poprzez ustawienie bitu 10 rejestru PCONP
    \item Ustawienie polaryzacji sygnału zegarowego SCK na wysoką poprzez ustawienie bitu 6 (CPOL) rejestru CR0
    \item Ustawienie czytania na rosnącym zboczu sygnału zegarowego poprzez wyzerowanie bitu 7 (CPHA) rejestru CR0
    \item Ustawienie preskalera tak, aby częstotliwość SCK była równa ok 1 MHz, poprzez wpisanie obliczonej wartości do bitów 8:15 (SCR) rejestru CR0
    \item Ustawienie liczby bitów tworzących bajt (ramkę) danych na 8 poprzez wpisanie wartości 7 do bitów 0:3 (DSS) rejestru CR0
    \item Ustawienie formatu ramki na format SPI poprzez wyzerowanie bitów 5 i 4 (FRF) rejestru CR0
    \item Wybranie trybu pracy 'master' poprzez wyzerowanie bitu 3 (MS) rejestru CR1
    \item Uruchomienie interfejsu poprzez ustawienie bitu 1 (SSE) rejestru CR1
\end{enumerate}
Po skonfigurowaniu i uruchomieniu interfejsu można pisać lub odczytywać dane. SSP posiada kolejkę FIFO zarówno dla danych wysyłanych jak i dla danych odczytywanych. Aby dodać kolejny bajt do kolejki (zostanie nadany kiedy tylko będzie możliwe) należy wpisać wartość do rejestru DR. Aby odczytać odebrany bajt należy odczytać zawartość rejestru DR.

Opisana powyżej konfiguracja jest zdeterminowana sterownikiem zewnętrznego wyświetlacza OLED, który na stałe ma ustawione pewne parametry interfejsu SPI (polaryzacja, faza), do których mikrokontroler (master) musi się dostosować.

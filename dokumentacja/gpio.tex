\subsection{GPIO}

\subsubsection{Opis funkcjonalności}

    Wejście i wyjście ogólnego przeznaczenia, wykorzystywane do komunikacji
    sprzętowej pomiędzy procesorem a urządzeniami peryferyjnymi, realizowane
    fizycznie jako piny.

\subsubsection{Zastosowanie}

    \paragraph{Wzmacniacz dla głośnika}

        W celu skorzystania z głośnika wymagane jest wcześniejsze włączenie
        wzmacniacza. W tym celu należy najpierw ustawić kierunek wyjściowy
        dla pinów P0.27, P0.28 i P2.13 poprzez ustawienie na wartość 1 bitów
        27 i 28 rejestru FIO0DIR oraz bitu 13 rejestru FIO2DIR.

        Następnie należy ustawić stan niski na wyżej wymienionych
        skonfigurowanych pinach. Dokonujemy tego wpisując wartość 1 do bitów
        27 i 28 rejestru FIO0CLR oraz bitu 13 rejestru FIO2DIR.

    \paragraph{Joystick}

        W celu konfiguracji pinów potrzebnych do obsługi joysticka, tj.
        P0.15, P0.16, P0.17, P2.3, P2.4, ustawiamy ich kierunek wejściowy
        poprzez wpisanie wartości 0 do bitów 15, 16 i 17 rejestru FIO0DIR oraz
        bitów 3 i 4 rejestru FIO2DIR.

        Aby odczytać wartość stanu joysticka, tzn. czy i w którą stronę został
        przyciśnięty, należy odczytać zawartość bitu [Y] rejestru FIO[X]PIN dla
        wspomnianych pinów P[X].[Y]. W przypadku wystąpienia na nim stanu
        niskiego wiemy, że joystick został wciśnięty w tym określonym kierunku.
        Kierunki i odpowiadające im piny definiujemy jako:
        \begin{enumerate}
            \item góra - P0.15
            \item dół - P2.3
            \item lewo - P0.16
            \item prawo - P2.4
            \item środek - P0.17
        \end{enumerate}
   